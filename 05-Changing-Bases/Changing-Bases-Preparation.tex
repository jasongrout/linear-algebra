\section{Preparation}

\noindent
This chapter covers the following ideas. When you create your lesson plan, it should contain examples which illustrate these key ideas. Before you take the quiz on this unit, meet with another student out of class and teach each other from the examples on your lesson plan. 


\begin{enumerate}
\item Explain how to describe vectors using different coordinate systems. Be able to change between different coordinate systems, and use these ideas to construct matrix representations of linear transformations.
\item Show that the null space (kernel), column space (image), and eigenspaces of a matrix (linear transformation) are vector subspaces. Be able to extend a basis for these spaces to a basis for the domain or codomain.
\item For square matrices, explain why similar matrices $B=P^{-1}AP$ represent the same linear transformation (just using a different basis). Explain why the determinant, eigenvalues, rank, and nullity of a linear transformation are not dependent upon the basis chosen.
\item Explain how to diagonalize a matrix, as well as explain when it is or is not possible. 
\end{enumerate}


Here are the preparation problems for this unit.

\begin{center}
\begin{tabular}{ll|l}
\multicolumn{2}{c}{Preparation Problems (\href{http://ilearn.byui.edu/bbcswebdav/institution/Physical\_Sci\_Eng/Mathematics/Personal\%20Folders/WoodruffB/341/5-All-Solutions.pdf}{click for handwritten solutions})}
%&
%Webcasts 
%(
%\href{http://ilearn.byui.edu/bbcswebdav/institution/Physical\_Sci\_Eng/Mathematics/Personal\%20Folders/WoodruffB/341/4-Linear-Transformations-videos.pdf}{pdf copy}
%)
\\
\hline\hline
Day 1& Schaum's 6.3, Schaum's 6.6, Schaum's 6.12, Schaum's 6.16
%&
%\href{http://ilearn.byui.edu/bbcswebdav/institution/Physical\_Sci\_Eng/Mathematics/Personal\%20Folders/WoodruffB/341/4-Linear-Transformations-video-01.wmv}{1},
%\href{http://ilearn.byui.edu/bbcswebdav/institution/Physical\_Sci\_Eng/Mathematics/Personal\%20Folders/WoodruffB/341/4-Linear-Transformations-video-02.wmv}{2},
%\href{http://ilearn.byui.edu/bbcswebdav/institution/Physical\_Sci\_Eng/Mathematics/Personal\%20Folders/WoodruffB/341/4-Linear-Transformations-video-03.wmv}{3}
\\ \hline
Day 2& 
Schaum's 9.1, Schaum's 9.6, Schaum's 9.18, 2b
%&
%\href{http://ilearn.byui.edu/bbcswebdav/institution/Physical\_Sci\_Eng/Mathematics/Personal\%20Folders/WoodruffB/341/4-Linear-Transformations-video-04.wmv}{4},
%\href{http://ilearn.byui.edu/bbcswebdav/institution/Physical\_Sci\_Eng/Mathematics/Personal\%20Folders/WoodruffB/341/4-Linear-Transformations-video-05.wmv}{5},
%\href{http://ilearn.byui.edu/bbcswebdav/institution/Physical\_Sci\_Eng/Mathematics/Personal\%20Folders/WoodruffB/341/4-Linear-Transformations-video-06.wmv}{6}
\\ \hline
Day 3& 
Schaum's 11.13, Schaum's 11.17, Schaum's 11.18, Schaum's 11.19
%&
%\href{http://ilearn.byui.edu/bbcswebdav/institution/Physical\_Sci\_Eng/Mathematics/Personal\%20Folders/WoodruffB/341/4-Linear-Transformations-video-07.wmv}{7},
%\href{http://ilearn.byui.edu/bbcswebdav/institution/Physical\_Sci\_Eng/Mathematics/Personal\%20Folders/WoodruffB/341/4-Linear-Transformations-video-08.wmv}{8},
%\href{http://ilearn.byui.edu/bbcswebdav/institution/Physical\_Sci\_Eng/Mathematics/Personal\%20Folders/WoodruffB/341/4-Linear-Transformations-video-09.wmv}{9}
\\ \hline
Day 4&
Lesson Plan,
Quiz, Start Project 
&
\\ \hline
\end{tabular}
\end{center}


The following homework relates to what we are learning in this unit.

\begin{center}
\begin{tabular}{|l|l|l|l|l|}
\hline
Concept&Where&Suggestions&Relevant Problems\\ \hline
Coordinates&Schaum's 6&1,3,4,6,9&1-10,22-28\\ \hline
Change of Basis &Schaum's 6&12,15,16,31&12-16,30-33\\ \hline
Matrix Representations (different bases)&Schaum's 9&16,18,19,44&16-20,42-45\\ \hline
Matrix Representations (same basis)&Schaum's 9&1,2,5,6,9,11&1-12,27-34,36,39\\ \hline
General Position& Here & 1,2& 1,2\\ \hline
Diagonalization&Schaum's 11&9,11,13,14,17,18&9,11-22,57,60-63\\ \hline
Visual illustrations&Here&TBA&TBA\\ \hline
\end{tabular}
\end{center}

Note that Schaum's does not use inverse matrices. Please take advantage of the fact that you can use inverse matrices to change coordinates. I have done by hand all the problems in Schaum's so that you can see how an inverse matrix yields the same solution. Please use these solutions. Also, remember that once you leave 2 by 2 matrices, please use your computer to find eigenvalues and inverse matrices.  You should practice finding the eigenvectors of some of the 3 by 3 matrices by hand.



