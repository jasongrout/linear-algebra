\section{Preparation}

\noindent
This chapter covers the following ideas. When you create your lesson plan, it should contain examples which illustrate these key ideas. Before you take the quiz on this unit, meet with another student out of class and teach each other from the examples on your lesson plan. 


\begin{enumerate}

\item Solve differential equations related to investing.
\item Use eigenvalues and eigenvectors to solve systems of differential equations.
\item Use generalized eigenvectors to find Jordan canonical form.
\item Find the matrix exponential of a square matrix, and use it to solve linear homogeneous ODEs. 
\item Give applications of systems of ODEs, as they relate to dilution problems. 

\end{enumerate}


Here are the preparation problems for this unit.

\begin{center}
\begin{tabular}{ll|l}
\multicolumn{2}{c}{Preparation Problems (\href{http://ilearn.byui.edu/bbcswebdav/institution/Physical\_Sci\_Eng/Mathematics/Personal\%20Folders/WoodruffB/341/6-Jordan-Form-Preparation-Solutions.pdf}{click for handwritten solutions})}
%&
%Webcasts 
%(
%\href{http://ilearn.byui.edu/bbcswebdav/institution/Physical\_Sci\_Eng/Mathematics/Personal\%20Folders/WoodruffB/341/4-Linear-Transformations-videos.pdf}{pdf copy}
%)
\\
\hline\hline
Day 1& 1a, 2a, 2b, 3a,
%&
%\href{http://ilearn.byui.edu/bbcswebdav/institution/Physical\_Sci\_Eng/Mathematics/Personal\%20Folders/WoodruffB/341/4-Linear-Transformations-video-01.wmv}{1},
%\href{http://ilearn.byui.edu/bbcswebdav/institution/Physical\_Sci\_Eng/Mathematics/Personal\%20Folders/WoodruffB/341/4-Linear-Transformations-video-02.wmv}{2},
%\href{http://ilearn.byui.edu/bbcswebdav/institution/Physical\_Sci\_Eng/Mathematics/Personal\%20Folders/WoodruffB/341/4-Linear-Transformations-video-03.wmv}{3}
\\ \hline
Day 2& 3b, 4dgh (they are all really fast), 5a, 5c
%&
%\href{http://ilearn.byui.edu/bbcswebdav/institution/Physical\_Sci\_Eng/Mathematics/Personal\%20Folders/WoodruffB/341/4-Linear-Transformations-video-04.wmv}{4},
%\href{http://ilearn.byui.edu/bbcswebdav/institution/Physical\_Sci\_Eng/Mathematics/Personal\%20Folders/WoodruffB/341/4-Linear-Transformations-video-05.wmv}{5},
%\href{http://ilearn.byui.edu/bbcswebdav/institution/Physical\_Sci\_Eng/Mathematics/Personal\%20Folders/WoodruffB/341/4-Linear-Transformations-video-06.wmv}{6}
\\ \hline
Day 3& 5e, 6a, 7a, 7b
%&
%\href{http://ilearn.byui.edu/bbcswebdav/institution/Physical\_Sci\_Eng/Mathematics/Personal\%20Folders/WoodruffB/341/4-Linear-Transformations-video-07.wmv}{7},
%\href{http://ilearn.byui.edu/bbcswebdav/institution/Physical\_Sci\_Eng/Mathematics/Personal\%20Folders/WoodruffB/341/4-Linear-Transformations-video-08.wmv}{8},
%\href{http://ilearn.byui.edu/bbcswebdav/institution/Physical\_Sci\_Eng/Mathematics/Personal\%20Folders/WoodruffB/341/4-Linear-Transformations-video-09.wmv}{9}
\\ \hline
Day 4&
Lesson Plan,
Quiz, Start Project 
&
\\ \hline
\end{tabular}
\end{center}


The accompanying problems will serve as our problems set for this unit.  You can find handwritten solutions online (just click the link at the top of the preparation table above). On problems where the system is not diagonalizable, the matrix $Q$ used to obtain Jordan form is not unique (so your answer may differ a little from mine in the intermediary steps, until you actually compute the matrix exponential $Qe^{Jt}Q^{-1}=e^{At}$).  

Try some of each of the problems in this unit. Make sure you do at least 7 each day.  
%\begin{center}
%\begin{tabular}{|l|l|l|l|l|}
%\hline
%Concept&Where&Suggestions&Relevant Problems\\ \hline
%\end{tabular}
%\end{center}




