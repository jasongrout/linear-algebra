\section{Notes for teaching}


\subsection{Ben Woodruff}

In class I talk about the following workflow which I plan to include in the book as well

$$\text{coordinates} \xrightarrow{(B')^{-1}} \text{vector} \xrightarrow{[T]} \text{vector} \xrightarrow{B'} \text{coordinates }$$

I tell the students that this chapter is all about writing vectors as linear combinations of specific vectors.
I emphasized the equations  $B [v] = v$ and $B[v] = B' [v']$ (since both equal $v$).  From these equations follows everything else in the entire unit. Some of your students will say ``haven't we already done all this in earlier chapters'' and the answer is ``essentially yes, but now we are focusing on the language.''

Schaum's does not use inverse matrices to do any computations, whereas the text uses inverses to motivate everything. In my mind, I always relate everything back to a standard basis before doing any computations.  Make sure you emphasize why solving $B[v]=v$ for $[v]$ using row reduction (as Schaum's does) is identical to using $[v]=B^{-1} v$.  Schaum's computes $B[v]$ and then solves $B'[v'] = (B[v])$ via row reduction, whereas I just immediately state $[v'] = B'^{-1} B[v]$.  The last time I taught this part of the text, I didn't allow them to use calculators.  I may consider changing that this time. The computations are just time intensive, even if you do a 2 by 2 example. 


